\documentclass{article}
\usepackage[utf8]{inputenc}
%\usepackage[T1]{fontenc}
\usepackage[left=0cm,right=0cm,top=0cm,bottom=0cm,bindingoffset=0cm]{geometry}
\usepackage[russian]{babel}
%\usepackage{amssymb,amsmath}

\begin{document}
\begin{tabular}{ |p{3.5cm}|p{3cm}|p{3cm}|p{3cm}|p{2cm}|p{2cm}|  }
\hline
\multicolumn{6}{|c|}{products продукты(устройства)} \\
\hline
Название & Описание & Тип данных & Ограничения & PK & FK\\
\hline
id                     &   % Название
                       &   % Описание
integer                &   % Тип данных
Unique, NOT NULL       &   % Ограничения
 +                     &   % PK
 +                     \\  % FK
\hline
manufacturer           &   % Название
Производитель          &   % Описание
varchar(20)            &   % Тип данных
                       &   % Ограничения
                       &   % PK
 +                     \\  % FK
\hline
start\_of\_production  &   % Название
Начало производства    &   % Описание
date                   &   % Тип данных
NOT NULL               &   % Ограничения
                       &   % PK
                       \\  % FK
\hline
end\_of\_production    &   % Название
Конец производства     &   % Описание
date                   &   % Тип данных
NOT NULL               &   % Ограничения
                       &   % PK
                       \\  % FK
\hline
model\_name            &   % Название
Название модели        &   % Описание
varchar(20)            &   % Тип данных
NOT NULL               &   % Ограничения
                       &   % PK
                       \\  % FK
\hline
\end{tabular}

\begin{tabular}{ |p{3.5cm}|p{3cm}|p{3cm}|p{3cm}|p{2cm}|p{2cm}| }
\hline
\multicolumn{6}{|c|}{manufacturers производители} \\
\hline
Название & Описание & Тип данных & Ограничения & PK & FK\\
\hline
name                   &   % Название
Название производителя &   % Описание
varchar(20)            &   % Тип данных
Unique, NOT NULL       &   % Ограничения
 +                     &   % PK
                       \\  % FK
\hline
foundation\_date       &   % Название
Дата основания         &   % Описание
date                   &   % Тип данных
                       &   % Ограничения
                       &   % PK
                       \\  % FK
\hline
\end{tabular}


\newpage

\begin{tabular}{ |p{3.5cm}|p{3cm}|p{3cm}|p{3cm}|p{2cm}|p{2cm}| }
\hline
\multicolumn{6}{|c|}{Template} \\
\hline
Название & Описание & Тип данных & Ограничения & PK & FK\\
\hline
\end{tabular}


\end{document}
